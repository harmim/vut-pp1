% Author: Dominik Harmim <harmim6@gmail.com>

%-------------------------------------------------------------------------------
%-------------------------------------------------------------------------------
%-------------------------------------------------------------------------------
%-------------------------------------------------------------------------------

\section{Introduction}

Bugs are an integral part of computer programs ever since the inception of the
programming discipline. Unfortunately, they are often hidden in unexpected
places, and they can lead to unexpected behaviour, which may cause significant
damage. Nowadays, developers have many possibilities of catching bugs in the
early development process. \emph{Dynamic analysers} or tools for \emph{automated
testing} are often used, and they are satisfactory in many cases. Nevertheless,
they can still leave too many bugs undetected, because they can analyse
only particular program flows dependent on the input data. An alternative
solution is \emph{static analysis} that has its shortcomings as well, such
as the \emph{scalability} on large codebases or considerably high rate of
incorrectly reported errors (so-called \emph{false positives} or \emph{false
alarms}).

Recently, Facebook introduced \emph{Facebook Infer}: a~tool for
creating \emph{highly scalable}, \emph{compositional}, \emph{incremental},
and \emph{interprocedural} static analysers. Facebook Infer has grown
considerably, but it is still under active development by many teams across
the globe. It is employed every day not only in Facebook itself, but also
in other companies, such as Spotify, Uber, Mozilla, or Amazon. Currently,
Facebook Infer provides several analysers that check for various types
of bugs, such as buffer overflows, data races and some forms of deadlocks
and starvation, null-dereferencing, or memory leaks. However, most importantly,
Facebook Infer is a~framework for building new analysers quickly and easily.
Unfortunately, the current version of Facebook Infer still lacks better
support for \emph{concurrency} bugs. While it provides a~reasonably advanced
\emph{data race} analyser, it is limited to Java and C++ programs only and
fails for C~programs, which use a~more \emph{low-level} lock manipulation.

In \emph{concurrent programs}, there are often \emph{atomicity requirements}
for execution of specific sequences of instructions. Violating these
requirements may cause many kinds of problems, such as unexpected behaviour,
exceptions, segmentation faults, or other failures. \emph{Atomicity violations}
are usually not verified by compilers, unlike syntactic or some sorts of
semantic rules. Moreover, atomicity requirements, in most cases, are not
even documented at all. So in the end, programmers themselves must abide by
these requirements and usually lack any tool support. Furthermore, in general,
it is difficult to avoid errors in \emph{atomicity-dependent programs},
especially in large projects, and even more laborious and time-consuming is
finding and fixing them.

In the thesis~\cite{harmimBP}, there was proposed the
\emph{Atomer}\,--\,a~static analyser for finding some forms of atomicity
violations implemented as a~module of Facebook Infer. In particular, the
stress is put on the \emph{atomic execution of sequences of function
calls}, which is often required, e.g., when using specific library calls.
The idea of checking atomicity of certain sequences of function calls is
inspired by the work of \emph{contracts for
concurrency}~\cite{contracts2017}. In the terminology of~\cite{contracts2017},
atomicity of specific sequences of calls is the most straightforward (yet
very useful in practice) kind of contracts for concurrency. The implementation
mainly targets C/C++ programs that use \emph{PThread} locks. Within this
project practice, Atomer was improved, extended, and other experiments were
performed. In particular, two new main features were introduced:
\begin{enumerate*}[label={(\roman*)}]
    \item
        support for \emph{C++ and Java locks},

    \item
        and distinguish \emph{multiple locks used}.
\end{enumerate*}
Moreover, working with \emph{sequences of function calls} was approximated
by working with \emph{sets of function calls} to make the solution more
scalable.

The development of Atomer has been discussed with developers of Facebook
Infer, and it is a~part of the H2020 ECSEL project Aquas. Parts of this
paper are taken from the thesis~\cite{harmimBP} and the
paper~\cite{excel2019FBInfer} written in collaboration with Vladimír Marcin
and Ondřej Pavela.

The rest of the paper is organised as follows. In Section~\ref{sec:fbinfer},
there is described Facebook Infer framework. Atomer is described in
Section~\ref{sec:atomer}, together with all the extensions and improvements
implemented within this project practice. Subsequently,
Section~\ref{sec:exp} discusses the experimental evaluation of the new
features and other experiments that were performed in this project.
Finally, Section~\ref{sec:conc} concludes the paper.

%-------------------------------------------------------------------------------
%-------------------------------------------------------------------------------
%-------------------------------------------------------------------------------
%-------------------------------------------------------------------------------

\section{Facebook Infer}
\label{sec:fbinfer}

\todo{foo}

%-------------------------------------------------------------------------------
%-------------------------------------------------------------------------------
%-------------------------------------------------------------------------------
%-------------------------------------------------------------------------------

\section{Atomicity Violations Detector}
\label{sec:atomer}

\todo{foo}

%-------------------------------------------------------------------------------
%-------------------------------------------------------------------------------
%-------------------------------------------------------------------------------
%-------------------------------------------------------------------------------

\section{Experimental Evaluation}
\label{sec:exp}

\todo{foo}

\cite{inferBiabduction}
\cite{contracts2017}
\cite{racerD}
\cite{AILatticeModelCousot}
\cite{dataflowAnalysisGraphReachability}
\cite{dataflowAnalysisApproaches}
\cite{excel2019FBInfer}
\cite{staticAnalysisMoller}
\cite{favStaticAnalysis}
\cite{favAI}
\cite{projectPracticeMarcin2018}
\cite{programAnalysisNielson}
\cite{AIBasedFormalMethodsCousot}
\cite{AIInNutshellCousot}
\cite{AICousotWeb}
\cite{wideningNarrowingCousot}
\cite{favLatticesAndFixpoints}
\cite{staticRaceDetectorTruePositive}
\cite{racerDOnline}
\cite{realWorldOCaml}
\cite{inferboOnline}
\cite{contracts2015}
\cite{controlFlowAnalysisAllen}
\cite{contract}
\cite{deadlockKroening}

%-------------------------------------------------------------------------------
%-------------------------------------------------------------------------------
%-------------------------------------------------------------------------------
%-------------------------------------------------------------------------------

\section{Conclusion}
\label{sec:conc}

This paper started by discussing a~\emph{static analysis} framework that uses
\emph{abstract interpretation}\,---\,\emph{Facebook Infer}\,---\,its features,
architecture, and existing analysers implemented in this tool. The major part
of the paper then aimed at the description of a~static analyser for detecting
\emph{atomicity violations}\,---\,\emph{Atomer}\,---\,implemented as a~module
of Facebook Infer and its extensions and improvements. Lastly, it is described
the experimental evaluation of the new features and other experiments performed
in this project practice, and it is discussed possible future work.

The original analyser works on the level of \emph{sequences of function calls},
but in this project, it was changed to work on the level of \emph{sets of
function calls}. The solution is based on the assumption that sequences
executed \emph{atomically once} should probably be executed \emph{always
atomically}. Within this project, two new main features were introduced:
\begin{enumerate*}[label={(\roman*)}]
    \item
        support for \emph{C++ and Java locks},

    \item
        and distinguish \emph{multiple locks used}.
\end{enumerate*}

The introduced extensions and improvements were successfully tested on
smaller \emph{hand-crafted} programs. It turned out that such innovations
enhanced the \emph{accuracy} and \emph{scalability} of the analysis. Moreover,
Atomer was experimentally evaluated on another software. Notably, it was
evaluated on \emph{open-source real-life Java programs}\,--\,\emph{
Apache Cassandra} and \emph{Tomcat}. Already fixed and reported real bugs
were successfully rediscovered. Nevertheless, so far, quite some \emph{false
alarms} are reported. However, a~result of the analyser can be used as an
input for \emph{dynamic analysis} which can determine whether the reported
atomicity violations are real errors.

Atomer again shows the potential for further improvements. The future work
will focus mainly on increasing the accuracy of the methods used by, e.g.,
distinguishing the \emph{context of called functions} by considering
\emph{formal parameters}, \emph{ranking} of atomic functions, or focusing on
\emph{library containers concurrency restrictions} related to method calls.
Further, it is needed to perform more experiments on \emph{real-life}
programs with an effort to find and report \emph{new bugs}.

The code of Atomer is available on GitHub as an \emph{open-source repository}.
The \emph{Pull Request} to the \texttt{master} branch of Facebook Infer's
repository is currently the work under progress. It is expected that work
on this project will continue not only within diploma thesis at FIT BUT.

%-------------------------------------------------------------------------------
%-------------------------------------------------------------------------------
%-------------------------------------------------------------------------------
%-------------------------------------------------------------------------------

\section*{Acknowledgements}
I~want to thank my supervisor Tomáš Vojnar for his assistance. Further,
I~would like to thank other colleagues from VeriFIT. I~would also like
to thank Nikos Gorogiannis from the Infer team at Facebook for useful
discussions about the development of the analyser. Lastly, I~thank for the
support received from the H2020 ECSEL project Aquas.
